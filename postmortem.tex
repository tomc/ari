\documentclass[tom-ari]{subfile}
\begin{document}
	
	\chapter{Post-Mortem Intro}
	
	This part of the document is to record hands and/or discussion which came up at the table for posterity.  I haven't fully decided how I want to organize this.  For the time being at least I am going to put each tournament into its own chapter, with sections for individual hands.  I can then use references in other sections of the document where the item in question came up.
	
	This may also help with crafting a to-do list, if we have items which came up that don't fit anywhere.
	
	\chapter{2018/2019 Monterey}
	
	\section{Double fit!}
	
	\vhand{akj,aj8,aj2,qj32} \vhand{2,kq762,87,akt54}
	\orauction{1c,1n,2c,2d,2h,2n,3c,3d,3s,4c,4n,5d,5s,6c,6h}
	
	When this hand came up, Tom was playing with Jenni.  (Ari was playing the same hands with Thuy.)  Jenni held the \cl1 hand, Tom held the 5-5 hand.
	
	4NT was intended as double RKC. Tom felt at the table that \di4 should have been RKC but Jenni meant 4NT as RKC, so answered 3 double KC.  (Jenni did mean 4NT as RKC.)  Then came the queen ask...but for what suit?  The \sp3 bid agreeing both suits is used to set up double KC, but also generally makes the second suit as the ``primary'' for purposes of expected trump suits, Q ask, etc.  Tom intended \cl6 to deny the \clubs{Q} and say nothing about the \hearts{Q}.  Jenni felt that there was likely a grand, but didn't want to pull the trigger with no info.
	
	Tom feels that the right answer to this problem is a counter-intuitive one:  do \emph{not} show the secondary fit.  You don't actually need to KC around the club suit.  While the \clubs{K} is valuable information, you can get that from RKC around hearts.  For example, \ldots4NT--\sp5--5NT--\cl6.  This is a difficult auction to see the trap without experience; every person shown this hand by Jenni has duplicated the \sp3 bid.  
	
	One of the things to remember about relay auctions is that they are a \emph{1-way} auction. We are taught from the time we are first starting to have a conversation with partner to find the best contract. Relays turn that on their ear -- the puppet master does not need to tell the puppet anything.  If you can get the info you need (such as the \clubs{Q}) via showing a double fit, then do it.  If confirming that will only muddy the auction, feel free to not show the secondary support and just continue with the one suit relay.
	
	This principle can be expanded to more than just confirming a double fit, of course. Relays can take some planning, especially when the relayer has multiple options or multiple questions they can ask.
	
	\section{Grand Ole Opry}
	\vhand{qt874,k9,j83,ak6} \vhand{akj96,a3,aq762,q}
	\orauction{1s,2d,2s,2n,3n,4n,5c,5n,7s}
	
	Another hand with Tom playing with Jenni. As it happens, this hand was against Ari and Thuy. Tom was Opener.
	
	The bids through 3NT seem normal enough, the question is how then to continue.  (There are other possible rebids than 2NT, such as \cl4 or \di3, but I think 2NT is better in the long run.)
	
	Jenni bidding 4NT then 5NT sounded to Tom like a very strong hand, but one either more balanced or more solid on the side. Tom thought that with 2 of the 3 side Ks, that was enough to bid grand.  However, today we really needed the \diamonds{K}. Tom strongly believes that a 5NT bid asking for Kings should be strong enough to support slam opposite 2 kings, at least in a pure power auction such as this one.
	
	I think instead of bidding 4NT directly, a \di4 cuebid first will help focus on that suit.  Now if the auction continues similarly we at least know what cards might be working.
	
\end{document}



