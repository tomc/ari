\documentclass[tom-ari]{subfile}
\begin{document}
	
\chapter{1 Major}
		
\section{Intro}
	
This section is new as of 22 Aug 2020 and agreed for the JLall.  It is a step based system, but different from what Ari and Tom played in the SF NABC in 2019.  
	
This spouts from a few different things:
	
\begin{itemize}
	\item Revisiting step based responses.
	\item Jenni/Bryan finding the previous \cl2 either/or TaJ unworkable
	\item Further examination of Moss/Grue methods
	\item Much soul searching
\end{itemize}
		
The idea is to have a step based responses as we did before, with keeping a separate TaJ-as-raise-only step.  The result is a mish-mosh of ideas we have explored or not.  I think this combined should work okay.

\section{Response Summary}

\begin{bidtable}{\orauction{1M}}
	+1 & Punt.  Like (Semi)Forcing NT.  Xfer responses to \sp1. \\
	+2 & GF, art. Typically denies 3+ card support\\
	+3 & TaJ, Limit+ \\
	+4 & Constructive to bad Invite in other major.  5+ cards, won't be 6+ in length and invitational.  (JS instead)  \\
	2M & Simple Raise \\
	JS & Nat Inv \\
	2NT & Mixed Raise - I talked with Jenni, she liked this to free up 1M--3M as pure weak. Also has symmetry with over opp's Dbl \\
	3M & Weak \\
	3NT & ??? \\
	DJS & Void Spl \\
\end{bidtable}

\section{+1: Punt}

The +1 bid is a punt to keep the auction alive, much like a forcing or semi-forcing NT.  \sp1--1NT is non-forcing and frequently passed when balanced.  (We don't need to worry about the 3 card LR.)

\subsection{\he1--\sp1}

\sp1 by responder does not deny spades, although it is unlikely to hold 5+ \sss~and a decent hand.  (See +4) A weak hand may hold any number of spades, as a \sp2 rebid is generally weak in most sequences.

Over this punt, we play transfer style responses.  The 1NT rebid is either \shape{5332} or 4+ \ccc.

Note that we do not have an ``Impossible \sp2''; \sp2 is just weak with spades.

\rem{Updated 11 Nov 2020}

\begin{bidtable}{\orauction{1h,1s}}
	1NT & 4+ \ccc~or balanced, NF \\
	\cl2 & 4+ \ddd \\
	\di2 & 6+ \hhh \\
	\he2 & Flannery; 4+ \sss \\
	\sp2 & \ccc, unwilling to play 1NT but not enough for a JS \\
	2NT & Spade reverse, 5-6 or better \\ 
	\cl3/\ddd & 5-5 Nat 13+ \\
	\he3 & 6+ Max; see below \\
\end{bidtable}

\begin{noted}[Abusing \di2]
	We do not have the 2NT toy to show a good 6--4 here, but it really isn't needed.  Opener can start with \di2 to show the 6+ card heart suit, then bid a 4 bagger next.\\
	\\
	Similarly, a good hand but bad suit 6+ \hhh hand might start with \di2 and see if partner can bid \he2 before committing to the 3 level. This implies that the direct \he3 is always a good suit.
\end{noted}	

After Opener's rebid, no special methods.  Bids which sound like sign off are sign off.

\subsection{\sp1--1NT}

1NT is NF, essentially a classic Semi-Forcing NT.  Pass is encouraged.  Typically bidding over 1NT is 4+ natural, or a non-minimum.  2NT is a good 6--4.

\section{+2: GF}

This is a \textit{non-relay} GF.  It can cover a wide range of GF hands, but it does not put us into a relay context automatically.  This allows for a (mostly) natural response structure by Opener, with a few tweaks (such as using 2NT to show extra length.)

Since the GF denies a primary fit, we can use retreating to 2M when available as a relay.  Otherwise most follow up sequences by responder are natural.

\subsection{\he1--1NT}

Very natural here.  Plenty of room to show hand type and preserve immediate 5-5 jumps.  JS just show shape, not extras.  

\he3 is normal 2/1 style jump, setting trump.  Instead of being limited to RKC or Qbids, we can immediately CIRKLE. (The last part is untested, but seems okay.) We keep 3NT as Non-Serious; \sp3 is CIRKLE in \hhh, \cl4/\ddd~ are serious Qbids.

\begin{bidtable}{\orauction{1h,1n}}
	\cl2 & Exactly 4 \ccc, can have long \hhh \\
	\di2 & Exactly 4 \ddd, can have long \hhh \\
	\he2 & Exactly 4 \sss, can have long \hhh; pushed down for space reasons \\
	\sp2 & Any 5332, 2NT asks LMH \\
	2NT & 6+ \hhh, non-solid \\
	3x & 5--5, any strength  (\sp3 is 5--6) \\
	\he3 & 0-1 loser suit, sets trumps. \sp3 over this is CIRKLE for \hhh, 3NT Non-S, 4m Serious \\
\end{bidtable}

\subsection{\sp1--\cl2}

This is very similar to the old structure of \cl2, with some small modifications. \di2 is no longer a pure waiting bid but actively showing a 4 card minor. \sp2 shows the 5332, 2NT shows extra \sss.

As per over \hhh, we use CIRKLE over the ``sets trump'' jump. 3NT retains Non-Serious, so \cl4 is CIRKLE and \di4/\hhh is a serious Q.

\begin{bidtable}{\orauction{1s,2c}}
	\di2 & Exactly 4 in either minor, can have long \sss \\
	\he2 & Exactly 4 \hhh, can have long \sss \\
	\sp2 & Any 5332, 2NT asks LMH \\
	2NT & 6+ \sss, non-solid \\
	3x & 5--5, any strength \\
	\sp3 & 0-1 loser suit, sets trumps. 3NT Non-S, \cl4 CIRKLE, 4Red Serious \\
\end{bidtable}

\subsection{Follow Ups}

2M is used as a shape relay.  If the suit is known below 2M (\he1--1NT--2x or \sp1--\cl2--\he2) then you jump directly into shapes in a length count up:  5, 6, 7+ with zooming. Relay for shortness (or zoom) VLH(N/B). (None only for 5, both only for 7+.  6-4 must have shortness but cannot have both).

If the suit is not known (\sp1--\cl2--\di2), step 1 is \ccc, then relay for 567.  Steps 2 and above are \ddd with the 567 count.

Over \he1--1NT--\he2 (showing \sss), \sp2 becomes the shape relay for 567.  Note this leaves us 3NT as the 4522 step!

Over \sp2 showing \shape{5332}, 2NT asks for the doubleton LMH.  

Once shape is known, if Responder can bid \cl4 then we are in a CIRKLE situation.  3NT is always a sign off bid; 3 level bids are ABCD CIRKLE as available, with \cl4 puppets to \di4 when 2+ CIRKLE bids remain. (If only 1, \cl4 is that CIRKLE).  \di4 by Responder is a puppet to \he4 for sign off.  Note that all of this is fairly typical CIRKLE, just spelled out here for clarity.

It is also quite likely that Responder will \textit{not} want to relay; if they choose any other actions than the ones listed above, they are natural in context.  2NT waiting, suits being natural, and so on.

\section{+3: TaJ}

3+ support, Limit Raise or better.  Note that we always have a min step below our trump suit, so we can limit our hands with that response and then make a game try with shapely hands.  This now is true for both majors, not just spades.

General steps are Min, Max, \"Uber (zoom). R+1 is relay for 5--4, 5--5, 6--4, 6+ short, 5 bal, 6 bal (zoom to Q).

\section{+4: Other Major}

\he1--\di2 and \sp1--\he2 both show 5+ cards in the other Major and limit the hand to less than GF values.  Most commonly this is Mixed strength and often will not reach game, but Opener is allowed to come to life with a fit.  Light Invitational hands are possible, but note that \sp1--\he2 is natural, so a very sound invite and only 5 \hhh{s} might choose to start with 1NT instead of \he2.  \he1--\di2 might be up to a real invite with 5, since this is a forcing call.

We still have \he1--\sp2 and \sp1--\he3 as natural and invitational with 6+, so this is less of a concern with real length in the OM.  

No special bidding here, 2 of either major suggests a contract.  2NT by Opener is Lebensohl; direct 3m is stronger but NF.

\section{Other}

Inv JS are still on for all suits.  2NT is now the Mixed Raise, with 3M being weak.  This also allows game tries over 2NT if wanted.

Direct double jumps are void splinters, as they were before in TaJ.  

3NT is not currently defined.  It was 17--18 bal before, but that really makes no sense. 1M-TaJ-2x-3NT is choice, so that isn't needed either.  Good 1/4 doesn't much make sense in a limited opener.  Open to suggestions.

\section{Passed Hand}

Everything is natural and NF. 1M-2M would be constructive, pass 1M with a weaker raise.

\begin{noted}[2NT best raise - 11 Nov 2020]
2NT is still a raise, in context it is the ``best'' raise, similar to \cl1-\di1-1M-2NT type sequences.  This came up (undiscussed but fielded!) at the table; it is unclear to Tom what the differences are between 2NT and 3M (which cannot be weak with no interference) but the underlying assumption is that 2NT > 3M > 2M. \\
\\
I think the comparison with the \cl1 sequence is a good one, so it makes sense to me to also play \cl3 as shortness ask.  Those notes currently say NLMH with 5+ trumps common; seems like a reasonable first pass at this bid as well.
\end{noted}

	  
\end{document}


