\documentclass[tom-ari]{subfile}
\begin{document}
	
	\chapter{Post-Mortem Intro}
	
	This part of the document is to record hands and/or discussion which came up at the table for posterity.  I haven't fully decided how I want to organize this.  For the time being at least I am going to put each tournament into its own chapter, with sections for individual hands.  I can then use references in other sections of the document where the item in question came up.
	
	This may also help with crafting a to-do list, if we have items which came up that don't fit anywhere.
	
	\chapter{2018/2019 Monterey}
	
	\section{Double fit!}
	
	\vhand{akj,aj8,aj2,qj32} \vhand{2,kq762,87,akt54}
	\nsauction[Jenni,Tom]{1c,1n,2c,2d,2h,2n,3c,3d,3s,4c,4n,5d,5s,6c,6h}
	
	When this hand came up, Tom was playing with Jenni.  (Ari was playing the same hands with Thuy.)  Jenni held the \cl1 hand, Tom held the 5-5 hand.
	
	4NT was intended as double RKC. Tom felt at the table that \di4 should have been RKC but Jenni meant 4NT as RKC, so answered 3 double KC.  (Jenni did mean 4NT as RKC.)  Then came the queen ask...but for what suit?  The \sp3 bid agreeing both suits is used to set up double KC, but also generally makes the second suit as the ``primary'' for purposes of expected trump suits, Q ask, etc.  Tom intended \cl6 to deny the \clubs{Q} and say nothing about the \hearts{Q}.  Jenni felt that there was likely a grand, but didn't want to pull the trigger with no info.
	
	Tom feels that the right answer to this problem is a counter-intuitive one:  do \emph{not} show the secondary fit.  You don't actually need to KC around the club suit.  While the \clubs{K} is valuable information, you can get that from RKC around hearts.  For example, \ldots4NT--\sp5--5NT--\cl6.  This is a difficult auction to see the trap without experience; every person shown this hand by Jenni has duplicated the \sp3 bid.  
	
	One of the things to remember about relay auctions is that they are a \emph{1-way} auction. We are taught from the time we are first starting to have a conversation with partner to find the best contract. Relays turn that on their ear -- the puppet master does not need to tell the puppet anything.  If you can get the info you need (such as the \clubs{Q}) via showing a double fit, then do it.  If confirming that will only muddy the auction, feel free to not show the secondary support and just continue with the one suit relay.
	
	This principle can be expanded to more than just confirming a double fit, of course. Relays can take some planning, especially when the relayer has multiple options or multiple questions they can ask.
	
	\rem{30 Nov 2019}

	Ari and I rebid this hand in a CIRKLE world yesterday, here was the auction:
	
	\nsauction[Ari,Tom]{1c,1s,1n,2c,2d,2s,2n,3c}
	
	... At this point, \he3 is no longer a double relay, but instead is CIRKLE in \heartsuit.
	
	\nsauction[Ari,Tom]{...,...,3h,4d,4s,4n,5c,5d,7n}
	
	\di4 showed 8 slam points, \sp4 was SQUARE for \heartsuit (\he4 would be sign off). 4NT stop showed 2 (AQ), \cl5 was SQUARE for \clubsuit, \di5 stop showed 2 (AK).  From there you can count 13 tricks.
	
	\section{Grand Ole Opry}
	\vhand{qt874,k9,j83,ak6} \vhand{akj96,a3,aq762,q}
	\nsauction[Jenni,Tom]{,1s,2d,2s,2n,3n,4n,5c,5n,7s}
	
	Another hand with Tom playing with Jenni. As it happens, this hand was against Ari and Thuy. 
	
	The bids through 3NT seem normal enough, the question is how then to continue.  (There are other possible rebids than 2NT, such as \cl4 or \di3, but I think 2NT is better in the long run.)
	
	Jenni bidding 4NT then 5NT sounded to Tom like a very strong hand, but one either more balanced or more solid on the side. Tom thought that with 2 of the 3 side Ks, that was enough to bid grand.  However, today we really needed the \diamonds{K}. Tom strongly believes that a 5NT bid asking for Kings should be strong enough to support slam opposite 2 kings, at least in a pure power auction such as this one.
	
	I think instead of bidding 4NT directly, a \di4 cuebid first will help focus on that suit.  Now if the auction continues similarly we at least know what cards might be working.

\ari{
The way I learned it, 5NT promises all keycards and the queen of trumps, allows partner to bid a grand if they have a source of tricks, and otherwise requests partner to show their cheapest king.  \ldots5NT--\cl6--\di6 would say "I understand you have the \clubs{K}, do you have the \diamonds{K} as well?" after which opener would make some bid above \sp6 to show the \diamonds{K}. On this hand, responder doesn't need to worry about opener having an undisclosed source of tricks because opener's shape is known to be \shape{5332}.}

\rem{I learned it exactly the opposite, where the \di6 bid in your example would show the King, not ask for it.}  

\ari{I'm not familiar with the agreement that 5NT requests partner to bid a grand with two kings. I'm open to playing that way but I don't think it's standard.}

\rem{I was speaking a little loosely, but to me 5NT first and foremost invites a grand slam.  I was a 5332 hand that didn't open 1NT and had a 4 control 13 count, it can't really get any better, so I accepted the invitation.  It's not that 5NT in all auctions asks responder to bid grand with 2 kings, but it is common.}

\ari{On this hand, I think responder should bid \cl4 over 3NT to see if partner can cuebid \di4 (which must be the \diamonds{K}) after which RKC would allow responder to easily bid the appropriate slam. I think \ldots3NT--\di4 would tend to be denying a club control.}

\rem{I think what you are saying is reasonable, however if you have a planned auction of suit{\ldots}4NT then it is common that you are focusing on the suit you bid. Clearly bypassing clubs than bidding 4NT over a signoff (not this hand) must have a club control.  Further, there is something to be said for the strong hand to adopt an Ace first cue style while the weaker hand does an up the line style.}

\chapter{JLall Nov 2020}

\section{How to encourage?}

	\begin{handdiagram}
	\boardnum{63}
	\dealer{s}
	\vul{n}
	\north[Tom]{kj75,j,k632,a542} 
	\east[Zia]{q964,a53,jt98,t6}
	\south[Ari]{2,qt9842,74,q987}
	\west[V. Gupta]{at83,k76,aq5,kj3}
	\end{handdiagram}

\auction{,,,2h,2n}

On the OL of the \hearts{j}, \hearts{a} from dummy, what card should Ari play?  At the table he played the \hearts{q} to try to clarify the solidity, but Tom interpreted this as encouraging but denying interest in the lowest suit; i.e., Att with secondary S/P connotations.  (vs. playing the 2).  

The only reason this holds is because so much about the suit is already known from the auction and declarer's play at trick 1.  The 10 could show that the highest card in the suit is the 10, the 9 might give away a Q9 holding, so that declarer has K10 situated over, and so on.  Both the Q and the 2 serve equally for attitude, which is why we can get away with a S/P signal here.

Note this cost zero tricks, just some added stress.  

\section{Cashout}

	\begin{handdiagram}
		\boardnum{60}
		\dealer{w}
		\vul{n}
		\north[Tom]{t,kt8,at6,ak9432}
		\east[Eddie Wold]{a73,aq7653,7,875}
		\south[Ari]{9854,94,k843,qj6}
		\west[Mike Levine]{kqj62,j2,qj952,t}
	\end{handdiagram}

\auction{1s,2c,2h,3c,3h,4c,4h}

West bid their hand strangely, but that led to part of the confusion.

OL was the \clubs{q} overtaken with a spade shift.  Eddie won on the board, Ari playing the \spades{9}. At the time, Tom interpreted this play as count in the spade suit.  3 rounds of hearts and Tom was in.  Cashed a club, cashed a diamond (3 from Ari), continued \ddd\ldots

While still down 1, we both felt this is a cash out situation we should get right.  Tom thought that it was likely that Ari had 4 clubs for the lightish raise, so thought it more likely that diamonds were cashing. He was also under the false impression that spades were blocked due to the earlier misinterpretation of the spade signal.  (Intended as S/P).  Ari admitted in brief discussion that there will be many times to encourage diamonds later and perhaps S/P isn't as necessary as count.  

The other table made a partscore so little difference in terms of IMPs today, but still worth noting and discussing.

\section{How bad is it?}

\begin{handdiagram}
	\boardnum{79}
	\dealer{s}
	\vul{n}
	\north{65,kj7,jt832,kqj}
	\east[Tom]{q98743,53,654,76}
	\south{k2,qt9642,aq,a43}
	\west[Ari]{ajt,a8,k97,t9852}
\end{handdiagram}

\auction{,,,1h,x,xx,3s,4h}

How bad is a favorable \sp3?  Tom thinks that it is extremely weak to be able to bid only \sp3 with 6+ instead of 4.  This likely indicates something like 0-3 HCP in \sss and 0--1 HCP outside.  On the actual deal \he4 cannot be set, but an extra trick was given up in the endgame when Ari shifted away from the \diamonds{k} because of Tom's T1 discourage in spades.  Tom was merely trying to deny the \spades{k} since it could be relevant for Ari to know if he needs to continue spades or shift away for tempo reasons.

1 IMP today, but the principle is important.  \sp2 is weak with 5, \sp3 (favorable) is extremely weak with 6+ and \sp4 is normal with a better hand with 6+ \sss.
	
\end{document}



