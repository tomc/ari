\documentclass[tom-ari]{subfiles}
\begin{document}

\chapter{1NT}

Our 1NT opening shows 14-16 HCP. 5422 hands and 5332 hands with a 5-card major are routinely opened 1NT.
5422 hands with a 5-card major and balanced hands with a 6-card minor can be opened 1NT as well.

\orauction{1n,?}
\begin{description}
  \item[\cl2] Stayman. Promises at least one 4-card major.
  \item[\di2] 5+ \heartsuit
  \item[\he2] 5+ \spadesuit
  \item[\sp2] Range ask. Includes hands with interest in \clubsuit.
  \item[2NT] Puppet stayman.
  \item[\cl3] 6+ \diamondsuit
  \item[\di3] 5+ \diamondsuit, 5+ \clubsuit, GF
  \item[\he3/\sp3] 1354/3154, GF. Singleton in the suit bid.
  \item[3NT] To play
  \item[\cl4] Gerber
  \item[\di4] Transfer to \heartsuit
  \item[\he4] Transfer to \spadesuit
\end{description}

\section{Stayman Sequences}

\orauction{1n,2c,?}
\begin{description}
  \item[\di2] No 4-card major
  \item[\he2] 4 \heartsuit, could have 4 \spadesuit
  \item[\sp2] 4 \spadesuit, typically denies 4 \heartsuit
\end{description}

\orauction{1n,2c,2d,?}
\begin{description}
  \item[\he2] Weak hand with both majors. Pass or correct.
  \item[\sp2] 5\spadesuit, invitational. All invites with 5\spadesuit ~go through this sequence. % ~ is a non-breaking space, forced the issue.
  % Other approaches such as \spades{5} might work. That's a BW thing, intended for cards/holdings but might work for inline cases like this too.
  % Enclosing inside of {} would probably also work.  I haven't tested. - Tom
  \item[2NT] Invitiational. Promises at least one 4-card major.
  \item[\cl3/\di3] 5+ m, GF.
  \item[\he3/\sp3] Smolen. 4M, 5+ OM, GF.
  \item[\cl4] Gerber
  \item[\di4] Delayed Texas. 4\spadesuit, 6\heartsuit
  \item[\he4] Delayed Texas. 6\spadesuit, 4\heartsuit
\end{description}

\orauction{1n,2c,2d,2s,?}
\begin{description}
  \item[P] Minimum, 2-3\spadesuit. With exactly 2, 2NT is an option as well.
  \item[2NT] Minimum, 2\spadesuit. 3m rebids by responder are natural. Still game invitational, but passable.
  \item[\cl3] Maximum with 2\spadesuit, GF. Responder may show 2 suited hands LMH via the next 3 steps.
  \item[\di3] Maximum with 3\spadesuit. Responder can bid 3NT to offer choice preferring NT, \sp3 to offer choice preferring spades, or \sp4 to sign off.
              Other bids are unusual, but possible with 2 suited hands.
  \item[\sp3] Minimum, 3\spadesuit. Better than pass, not enough to commit to game.
  \item[3NT] Probably a 2-card maximum that forgot to bid \cl3.
\end{description}

\orauction{1n,2c,2M,?}
\begin{description}
  \item[\sp2] (Over \he2). 5\spadesuit, invitational. Same followups as over 1NT-\cl2-\di2-\sp2.
  \item[2NT] Invitiational. Promises 4 cards in OM.
  \item[\cl3] 5+ \clubsuit ~OR 5+ \diamondsuit, GF. \di3 asks for the minor, LH. Other bids are natural, including bidding the other major to confirm a fit there.
  \item[\di3] Artificial, confirms a fit in M, typically no shortness. Opener can bid 3M to suggest playing in M, 3NT to suggest playing there, or 4M to insist on M, denying slam interest. Other suit bids are cuebids with hands well suited for slam.
  \item[3M] Invitational
  \item[3OM] Unspecified splinter. Next step asks, LMH.
  \item[3NT] To play
  \item[\cl4] 4M, 6OM, slam try. Opener's 4OM rebid is an offer to play.
  \item[\di4] RKC for M
  \item[4NT] Quantitative
\end{description}

\section{Jacoby Sequences}

\end{document}
