\documentclass[tom-ari]{subfile}
\begin{document}
	
	\chapter{1 Major}
	
	\section{Intro}
	
	12 Jan 2020: Reverted to what we decided by the end of SF. This reflects playing \he1--\sp1 as natural rather than a relay.
	
	\section{Response summary} 

	Bids where they differ will be noted, but for the most part they retain the same meaning for both \he1 and \sp1.	
	
	\begin{bidtable}{\orauction{1M}}
		\sp1 & ``Standard'', 4+ cards forcing 1 round. \\
		1NT & Semi-Forcing NT \\
		\cl2 & 1 of 3 hand types: LR+, GF \clubsuit, GF balanced. Respond as per TaJ. \\ 
		\di2 & Natural GF. Allowed to bid either \cl2 or \di2 with 5332 no fit.\\
		\he2 & (After \sp1) Natural GF\\
		\sp2 & (After \he1) Natural, Inv \rem{Was mixed under the artificial \sp1 world, but changing to inv to match other jumps.} \\
		2NT & 5-5 minors, Mixed+.  Open to other meanings if useful. Strength open to discussion. \\
		3NT & Bal with M, choice of games.  No slam interest. \\
		R & Simple raise, less than inv. \\
		3x & Natural, Inv \\
		3M & Mixed \\
	\end{bidtable}

	\section{Semi-Forcing NT}

	2 minor bids are generally natural 4+ cards. Can be short if no other bid appeals and too strong to pass. (i.e., \di2 on \exactshape{5332}). Rebidding the other major is always natural; non-reverse can be any strength, reverse shows \shape{56xx} non-min, like a jump shift. 
	
	3 of a new suit is a 5--5 jump shift, non-forcing but maximum. 6--4 jump shifts are handled via 2NT. Over that, \cl3 is pass or correct, \di3 asks LMH (GF).
	
	Simple rebids and jump rebids in the major are both standard, 6+ NF.
		
	\section[2C TaJ]{\cl2 TaJ}
	
	This is the primary relay step.  It is inheriting the response structure from TaJ, but doesn't guarantee a fit the way TaJ does.  It is intended to be general purpose, which makes the bid mostly fall into 1 of 3 hand types:  a classic TaJ hand (LR+), any GF bal or other that feels like relaying, any GF in clubs.  The club hand can be shown by breaking relay and bidding \cl3 or \textbf{\textit{or higher. (i.e., \di3 would be primary clubs and secondary diamonds.)}} The only danger is that partner bids an \"Uber step 2NT or higher, but we are likely in a slam position so it may not matter much.
	
	\reversemarginpar
	\marginpar{\includegraphics[width=.8\marginparwidth]{new.png}\hfill} 
	
	The responses are as per TaJ, shifted down.  1st step is any min, 2nd step is any medium ``I would accept a LR'' type hand, steps above that are \"Uber good and break into TaJ steps.
	
	Over the more common +1/+2 responses, retreating to 2M is the weakest action, showing the LR type hand.  In the case of +2, this becomes the ``Are you sure?'' bid instead of raising to the 3 level.
	
	The cheapest bid that isn't 2M is the TaJ relay, as per 2NT in the old system.  Once we have relayed fully, we can advance into CIRKLE/SQUARE. 3NT is never an asking bid in relay auctions, it is always an attempt to play there. As in other auctions, when space is tight for CIRKLE then \cl4 becomes a puppet for any remaining CIRKLEs that can't be bid directly, \di4 is a puppet to sign off anywhere, and direct game bids above 3NT are NF but encouraging.

	A relay-then-3NT auction loses it's previous meaning of a Choice of Games, so the direct 1M--3NT responses have become that. The new meaning of relay-then-3NT is simply no slam interest opposite partner's strength, NF.
	
	\textbf{\textit{Added based on discussion with Jenni and Bryan:  2NT after the response (when not the relay) is a Woolsey style game try, asking for the lowest suit that Opener would accept a help suit GT.  The only time 2NT is a relay is \sp1--\cl2--\he2, so we should be in a more-or-less GF auction anyway, so a Woolsey GT isn't needed.}}
	
	\reversemarginpar
	\marginpar{\includegraphics[width=.8\marginparwidth]{new.png}\hfill} 
	
	\rem{Does it make sense to have a bail option to go straight to CIRKLE rather than needing to fully TaJ out? I can imagine, for example, hearing that partner is 54xx and not caring about the details, or even uncovering the 4 card suit and then bailing. Sample auction might be something like \sp1--\cl2--\di2--\he2--\sp2--2NT--\cl3--\he3.}
	
	\section[2D Nat GF]{\di2 Nat GF}
	
	GF with 5+ diamonds. You may choose to relay instead of showing diamonds, this is up to the responder. 2NT is the default response with nothing descriptive to say.
	
	\rem{Some people I play with like inverting 2M an 2NT here, making 2NT show 6+M and 2M be waiting. Thoughts?}
	
	\section[1S--2H]{\sp1--\he2}
	
	GF with 5+ hearts. You may choose to relay instead of showing diamonds, this is up to the responder.
	
	\rem{I would like to add some structure here. I played a structure with Steve Beatty in a standard context where the simple rebids by opener were \sp2 denying 3 hearts and 2NT showing 3 hearts. Direct 3 bids can be used to show special hand types, such as 5-5s, 4 hearts, etc. There were considerations he and I needed to worry about because of standard that don't apply to us, which should make it even simpler. Everything in the PROPOSAL section will be considered a non-agreement until I hear otherwise.}
	
	\rem{Altneratively, see the \di2 section above about \sp2/2NT inversion. I could even see combining the ideas somehow.}
	
	\Huge{\color{red}PROPOSAL}
	
	\normalsize
	
	Work in Progress
	
	\begin{bidtable}{\orauction{1s,2h}}
		\sp2 & 0--2 hearts, denies 5--5 or 6--4. \\
		2NT & 3 hearts, usually exactly. \\
		\cl3/\di3 & Natural, 5--5 or 6--4. \\
		\he3 & 4+ hearts maximum, +1 relay for NLH shortness or +2 for CIRKLE in \heartsuit. \\ 
		\sp3 & 0--1 loser spades, sets trumps. \\
		3NT--\di4 & 4+ hearts minimum, NLH shortness. \rem{Fast arrival} +1(not \he4) is CIRKLE. \\
		\he4 & \exactshape{5422} with no interest in 3NT. Typically a ``picture'' hand, nothing in the minors. \\
	\end{bidtable}

	After \sp1--\he2--\sp2, 2NT is a shape relay. Typically this is a more balanced heart hand, but can be shapely with extra strength. (i.e., normal relay considerations). Other bids are natural. (\he3 promises 6+.)
	
	\begin{bidtable}{\orauction{1s,2h,2s,2n}}
		\cl3/\di3 & Natural, 5/4. +1 agrees the minor and relays for NLH, then CIRKLE. +2 ``Optional CIRKLE'' in hearts: first step is 0--1 hearts, other steps are CIRKLE for \heartsuit. Over +1, \cl4 is CIRKLE anyway. \\
		\he3 & 6+ \spadesuit, max. \sp3 relay for NLMH, 3NT NF, \cl4 CIRKLE \heartsuit, \di4 CIRKLE \spadesuit. 4M NF. \\
		\sp3 & 6+ \spadesuit, min. 3NT relay NLMH, \cl4 CIRKLE \heartsuit, \di4 CIRKLE \spadesuit. 4M NF. We lose the ability to play 3NT here. \\
		3NT & \exactshape{5233} \\
	\end{bidtable}
		
	
	\section{2NT}
	
	Both minors?  Really a place holder agreement, I'm open to anything. Mixed+? Other meanings might be better still, but I'm currently at a loss for finding a ``good'' use for this bid. Could be CIRKLE for M I suppose.
	
	\section{3NT}
	
	Choice of games, promises 3 card support for M.
	
	\section{Jump Shift}
	
	Natural, game invitational. 3M rebid by opener NF.
	
	\section{Double Jump Shift}
	
	Void Splinter. Singleton splinters start with TaJ. (\cl2)
		
\end{document}


