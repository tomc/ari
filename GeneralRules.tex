\documentclass[tom-ari]{subfile}
\begin{document}
	
\chapter{General Rules}

Some rules in no particular order.

\begin{itemize}[]
\item \textbf{Relays Off} Most relay auctions bail in competition, we don't tend to try to continue to relay when the opponents are interfering.
\item \textbf{Fast arrival} When we are forced to a certain level (say after a cuebid), bidding that level is the weakest action. This is true even over something like a double, where pass is more encouraging than retreating.
\item \textbf{Jumps} Most single jump shifts in comp are fit showing. Most jump raises are weak. Double jumps where available are splinters, but are lower precedence than Fit when only 1 jump available. Double jump cuebid is mixed when available.
\item \textbf{Cuebids} Our direct cuebids in response to partner's call generally show support, although it is possible that some one off cases may exist where you need to force with no good options.  Delayed cuebids are probing / generally ask for stoppers. When 2 opponent suits are available, we cuebid what we have not what we are looking for.
\item \textbf{2x Cuebid} As a psyche protection (primarily in \cl1 auctions), if we cuebid the opps suit twice that is \textit{natural}. This has come up in play and has proven useful.
\end{itemize}
	
\end{document}