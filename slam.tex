\documentclass[tom-ari]{subfile}
\begin{document}
\chapter{Slam Bidding}

\section{RKC}

In general, we play a 1430 style of RKC with ``Redwood'':  +1 RKC for minors, 4NT RKC for majors.

Over the Q ask, we play new suits as showing both the Q and the lowest ranking side K.  5NT when there is room to show all the Ks in undefined, although Ari bid it at the table intending it as 2 side Ks.  To Tom, 5NT is a substitute for a K when needed but unclear what the meaning is/should be otherwise.

\rem{Note!}

One possibility that I know some players do is a SQUARE-like response structure, with go/stop for the trump Q then Ks UTL. We perhaps should explore this further.

\subsection{Exclusion}

Exclusion is always 0314, the only non-1430 RKC we play.

\subsection{Preempt KC}

Auctions such as weak 2 - \cl4, responses are fairly typical:  0, 1 w/o Q, 1 w/, 2.  Note we won't have 2 with so no steps beyond this are required.

\subsection{Showing Voids}

I do not believe we have any firm agreements as to how to show a void over 4NT RKC.  There are a number of schemes, non of which are all that great.  Open to suggestions.

\section{CIRKLE/SQUARE}

See p. \pageref{CIRKLE} and p. \pageref{SQUARE}.

\section{Cuebid Style}

Most cuebids are loose as to \first control strictly vs. \first/~\second.  In situations where a very strong hand is cuebidding it is generally assumed to be first, whereas a weaker hand cuebidding can be anything.

\section{Other}

3NT is Non-Serious in many auctions, over which all cuebids are \first/~\second style.  Note that we never use other bids for Non-S, only 3NT.
\end{document}