\documentclass[tom-ari]{subfile}
\begin{document}
	
	\chapter[1C]{\cl1}
	
	\cl1 is our general forcing opening, showing roughly 17+ HCP balanced or 16+ unbalanced.  The new ACBL convention charts allow for as low as 14 HCP for a forcing bid so long as it meets the rule of 24.  (i.e., 10 cards in 2 suits).  I don't know how much we will take advantage of the new rule, but it is worth noting.
	
	Like all strong club systems, the strength of the system lies not within the \cl1 opening itself but rather allowing for all other openers to be limited.  While we have some special sequences which allow for \"{u}ber max in TaJ sequences, most of our simple 1 level openings are going to be lighter than most people open.  All of our invitational and competitive decisions revolve around that fact.  This can be a factor in deciding between \cl1 and 1M, for example.
	
	As a first pass for Zar, most \cl1 openers should be 32+ Zar.  (This is simply doubling 16 and using that as a first guess.)
	
	\rem{As I noted in the mad science section, I'm open to changing the response structure if we feel it will better support good relays.}
	
	\orauction{1c,?}
	
	\begin{description}
		\item[\di1] Negative, typically 0-7 HCP.  Can include bad 8s.  Roughly up to 19 Zar, based on 52=game and 32=\cl1.
		\item[\he1] 5+ \spadesuit, GF.  Generally 8+ HCP, although good 7 is okay. A+K is usually upgraded.  Note that \shape{5332} with an A and K is 21 Zar, so this fits the scheme well.
		\item[\sp1] Clubs or balanced (or limited \shape{4441}), GF.  Standard evaluation, although only the Clubs hands are unlimited.  Balanced is roughly 8-12 HCP.
		\item[1NT] 5+ \heartsuit, GF
		\item[\cl2] 5+ \diamondsuit, GF
		\item[\di2] Semi-positive transfer.  6+ \heartsuit, roughly 3-6 HCP.  Not GF
		\item[\he2] Semi-positive transfer, 6+ \spadesuit, roughly 3-6 HCP.  Not GF
		\rem {\item[Note] While I like the semi-positives, I recognize that they are taking up valuable bidding space that can likely be better utilized.}
		\item[\sp2] Big balanced.  No 5 card suit, 13+ HCP or 5+ controls.  In Zar this equates to 27+.
		\item[2NT] \exactshape{1444} 13+ HCP
		\item[\cl3] \exactshape{4441} 13+ HCP
		\item[\di3] \exactshape{4414} 13+ HCP
		\item[\he3] \exactshape{4144} 13+ HCP
		\item[\sp3] \rem{**NEW**} AKQxxxx, any suit.  This is different from older versions which had the \exactshape{1444} repeated.
		\item[3NT--\he4] 8 card suit transfers, very weak.  QJxxxxxx is expected maximum.
		\item[\sp4 \& up] Undefined
	\end{description}
	
	\section[1C--1D]{\cl1--\di1}
	
	\di1 is the general negative bid.  With the exception of the semi-positive transfers, this is the only bid which does not set up a GF auction.
	
	Crack theory time:  one of the ideas I am considering is what would happen in a \di1 ``waiting'' style instead of a ``negative''.  If \di1 could include some minimum balanced GFs, it could make some auctions easier.
	
	\orauction{1c,1d,?}
	
	\begin{description}
		\item[\he1] 4+ \heartsuit, can have a longer minor, 1RF.  Unbalanced or semi-balanced.
		\item[\sp1] 4+ \spadesuit, can have a longer minor, 1RF.  Unbalanced or semi-balanced.
		\item[1NT] 17--19 bal, can have 5CM or 6Cm.  \shape{5422} also possible.
		\item[2m] Nat NF.  Denies 4CM.  Typically 6+ cards and unbalanced.
		\item[\he2] Kokish Relay.  Forces \sp2, Either GF with hearts or GF Bal.
		\item[\sp2/3m] GF Nat, typically 1 suited.
		\item[2NT] 20--21 bal
		\item[\he3 \& up] Undefined, although game bids are simply to play.

	\end{description}
	
\end{document}




