\documentclass[tom-ari]{subfile}
\begin{document}
	
\chapter{General Rules}

Some rules in no particular order.

\begin{itemize}[]
\item \textbf{Relays Off} Most relay auctions bail in competition, we don't tend to try to continue to relay when the opponents are interfering.
\item \textbf{Fast arrival} When we are forced to a certain level (say after a cuebid), bidding that level is the weakest action. This is true even over something like a double, where pass is more encouraging than retreating.
\item \textbf{Jumps} Most single jump shifts in comp are fit showing. Most jump raises are weak. Double jumps where available are splinters, but are lower precedence than Fit when only 1 jump available. Double jump cuebid is mixed when available.
\item \textbf{Cuebids} Our direct cuebids in response to partner's call generally show support, although it is possible that some one off cases may exist where you need to force with no good options.  Delayed cuebids are probing / generally ask for stoppers. When 2 opponent suits are available, we cuebid what we have not what we are looking for.
\item \textbf{2x Cuebid} As a psyche protection (primarily in \cl1 auctions), if we cuebid the opps suit twice that is \textit{natural}. This has come up in play and has proven useful.
\end{itemize}

\section{General defenses}

\begin{itemize}[]
\item \textbf{2 Suiter, known} We play lower cuebid for our lower suit, higher for higher. Double is a good hand, with a second double being penalty.

In general the cuebid is the stronger action; the only exception is when the high cuebid is below our low suit, then the bidding the high suit is stronger than the cuebid.  In common practice, the only time this is relevant is over \he1--(2NT), where \di3 showing spades but not strong (less than GF) allows for a \he3 rebid (NF). \sp3 in that instance is forcing.	

\item \textbf{2 Suiter, 1 unknown} We treat this auction as if they had bid the one known suit. Cuebid is support for opener, new suits forcing, etc.

\item \textbf{2NT} In competition, 2NT might mean many different things depending on context. When the bidder is ``forced'' (i.e., responding to a takeout style double) the default meaning is Scrambling. When 2NT is a free bid it is most commonly Good/Bad; one notable exception is when 2NT is the first bid by responder, in which case it is natural except where otherwise defined. (i.e., 1M-Dbl).

One other possibility in ``Good/Bad'' sequences where Good isn't possible (hand already limited severely): secondary but higher ranked suit. These need to be better defined. 
\end{itemize}

\end{document}