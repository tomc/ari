\documentclass[tom-ari]{subfile}
\begin{document}
	
	\chapter[1C]{\cl1}
		
	\section{Description \& Response Summary}
	
	\cl1 is our general forcing opening, showing roughly 17+ HCP balanced or 16+ unbalanced.  The new ACBL convention charts allow for as low as 14 HCP for a forcing bid so long as it meets the rule of 24.  (i.e., 10 cards in 2 suits).  I don't know how much we will take advantage of the new rule, but it is worth noting.
	
	Like all strong club systems, the strength of the system lies not within the \cl1 opening itself but rather allowing for all other openers to be limited.  While we have some special sequences which allow for \"{u}ber max in TaJ sequences, most of our simple 1 level openings are going to be lighter than most people open.  All of our invitational and competitive decisions revolve around that fact.  This can be a factor in deciding between \cl1 and 1M, for example.
	
	\Huge{\color{red}ALPHA}
		
	\normalsize
	
	\rem{(8 Jan 20) Modifying the previous BETA notes, these are ideas I've bounced off Jenni. It's a combination of the old and new ideas; less exact shape relays, but more strength bids and CIRKLE tie ins. Writing this up to kick off discussion, can always be reverted in GIT if need be.}
	
	\begin{bidtable}{\orauction{1c}}
		\di1 & No change, still negative. \\
		\he1 & (UPH) Extra values w/o hearts \\
		\he1 & (PH) Clubs. \\
		\sp1 & 5+ Hearts unbal.  This is unlimited, unlike the other GF responses.  TaJ has the extra values step. \\
		1NT & 5+ Spades unbal, no extras.  \cl2 TaJ, \sp2 is natural clubs. \\
		\cl2 & (UPH) Clubs or Diamonds.  \di2 asks suit with LH \& zoom into TaJ with \diamondsuit\\
		\cl2 & (PH) Diamonds. \di2 TaJ. \\
		\di2 & 4 card \heartsuit ~bal\\
		\he2 & 4 card \spadesuit ~without 4 \heartsuit, bal \\
		\sp2 & No 4 card major, balanced \\
		2NT & \exactshape{1444} \\ 
		\cl3 & \exactshape{4441} (bid sing)\\
		\di3 & \exactshape{4414} (bid sing)\\
		\he3 & \exactshape{4144} (bid sing)\\
		\sp3 & ``Gambling'' hand, AKQxxxx or better. Typically no side cards. \\
		3NT--\he4 & 8+ card transfers, bust hand. No A or K. \\	
	\end{bidtable}



	\section[1C--1D]{\cl1--\di1}
	
	\di1 is the general negative bid.  This is the only bid which does not set up a GF auction.
	
	Meckwell style rebids except 2NT is 20--21.
	
	\begin{bidtable}{\orauction{1c,1d}}
		\he1 & 4+ \heartsuit, can have a longer minor, 1RF.  Unbalanced or semi-balanced. Systemic rebid with 4=4=(4-1)\\
		\sp1 &  4+ \spadesuit, can have a longer minor, 1RF.  Unbalanced or semi-balanced.\\
		1NT & 17--19 bal, can have 5CM or 6Cm.  \shape{5422} also possible.\\
		\cl{2}/\di2 & Nat NF.  Denies 4CM.  Typically 6+ cards and unbalanced.\\
		\he2 & Kokish Relay.  Forces \sp2, Either GF with hearts or GF Bal.\\
		\sp2/\cl3/\di3 & GF Nat, typically 1 suited.\\
		2NT & 20--21 bal\\
		\he3 \& up & Undefined, although game bids are simply to play.		\\
	\end{bidtable}

	\begin{bidtable}{\orauction{1c,1d,1h}}
		\sp1 & 4+ \spadesuit, any strength.  Typically fewer than 4 \heartsuit.  Most rebids are natural NF, minor suits can be canap\'e. 2NT is an artificial big canap\'e (6+ m) 1RF.  Jumps encouraging but NF with jump shifts being 5--5. \\
		1NT & 0--5, no 4CM.  Rebids as per over \sp1, except \sp2 is a natural reverse and 1RF.\\
		\cl2 & 0--2 \heartsuit, 5+ to 7.  \di2 is waiting and scrambling, \he2 is natural and NF. Other GF.  2NT is a non-canap\'e GF, 3m is canap\'e.\\
		\di2 & Exactly 3 \heartsuit, 5+ to 7. \he2 NF, \he3 Inv.  2NT GF asking for shortness NLMH. Other 1RF, usually canap\'e.\\
		\he2 & 4+ \heartsuit, minimum. New suits are game tries, 2NT asks shortness NLMH.\\
		2NT & Best raise, nearly GF.  5+ \heartsuit ~common, \cl3 asks for shortness NLMH.\\
		JS & 6+ nat, 5+--7\\
		DJS & Splinter with exactly 4\heartsuit		\\
	\end{bidtable}
			

	
	\begin{bidtable}{\orauction{1c,1d,1s}}
		 & As per over \he1, except \he2 shows 5+ \heartsuit, 5+--7.
	\end{bidtable}
	
	\begin{bidtable}{\orauction{1c,1d,1n}}
		& 17--19, systems on as per 1NT opening.
	\end{bidtable}

	\begin{bidtable}{\orauction{1c,1d,2m}}
		& Natural, NF, denies 4CM.  No special follow ups. Jump Shift is a splinter.
	\end{bidtable}

	\begin{bidtable}{\orauction{1c,1d,2h,2s}}
		& \he2 is Kokish, forces \sp2.  Either \heartsuit ~or bal, GF.  No agreements about bids other than \sp2 by responder. \\
		2NT & GF Balanced.  Systems on as per 2NT opener. \\
		\cl3 & \heartsuit ~\& minor, \di3 for LH. \\
		\di3 & One suited \heartsuit \\
		\he3 & \heartsuit ~\& \spadesuit \\
		Other & ? \rem{Self Spl?} \\
	\end{bidtable}
	
	\begin{bidtable}{\orauction{1c,1d,2n}}
		& 20--21, as per 2NT opener
	\end{bidtable}

	\begin{bidtable}{Other Rebids}
		& Other jumps are natural GF.  No special agreements other than \ldots\sp2--2NT is a spade raise, with \sp3 being the more waiting nothing-to-say type bid. Typically bal or near bal, 1--2 \spadesuit.
	\end{bidtable}

	For responses at \di2 or above, opener may skip the relay step to zoom into CIRKLE directly. The known length suit is ``A'' for this purpose, other suits are BCD in game order.  (For \sp2, \heartsuit is A and \spadesuit is B, per normal rules.)

	\section[1C--1H]{\cl1--\he1}
	
	Good hand by unpassed hand, clubs by passed hand.
	
	\subsection{UPH}
	
	By an unpassed hand, this shows the ``extra values'' step we previously used mid-relay. Generally a good 12 or higher, although 5 control 11 counts (AAK) also are treated as extras. 12 can go high or low, 13 is always high.
	
	We avoid this bid with primary hearts. \sp1 over \cl1 therefore still has an extra value step in TaJ. We can of course still have a heart suit if it ``secondary'', i.e. 5 spades or 6+ minor. Nothing is perfect.
	
	\begin{bidtable}{\orauction{1c,1h}}
		\sp1 & Waiting bid, all of responders bids are as per direct over \cl1, just stronger. Most common bid by opener, retaining captaincy. \\
		1NT & Hearts, inverted captaincy. Typically responder will bid \cl2 to get TaJ from opener or break relay and bid naturally. \\
		\cl2 & Spades, inverted. TaJ \\
		\di2 & Clubs, inverted. TaJ \\
		\he2 & Diamonds, inverted. TaJ \\
	\end{bidtable}
	
	The inverted responses show an unbalanced hand by opener and a desire to describe rather than ask. This may especially make sense with hearts, as responder will be declaring that strain. Obviously possible with any suit.
	
	Examples:
	
	\orauction{1c,1h,1s,2d} 
	
	Balanced hand with exactly 4 hearts, extra values.
	
	\orauction{1c,1h,1n,2c,2h}
	
	1NT showed hearts, \cl2 TaJ, \he2 shows some 5-5 hand. If responder breaks relay, it is natural with our normal swaps where applicable. In this example, if responder bids \he2 over 1NT he would show clubs.
	
	\subsection{PH}	
	
	\he1 shows 5+ \clubsuit, GF.  \sp1 is TaJ, \cl2 is \heartsuit (normal inversion), other natural. 
	
	\rem{There is an argument that 1NT should be TaJ to pick off the NTs, but I think that having that one off exception is too much memory work for the minimal gain.}
	
	\section[1C--1S]{\cl1--\sp1}
	
	5+ \heartsuit, GF. By an UPH this bid still has the extra values step, as we avoid bidding \he1 to try to avoid wrong siding and to allow the symmetry for responses. This is the only positive response which retains the extra values step for TaJ.
	
	1NT is TaJ, other bids are natural. There is no need for a swap here. \he2 is undefined.
	
	\section[1C--1NT]{\cl1--1NT}
	
	5+ \spadesuit, limited GF. \cl2 TaJ, \sp2 is clubs. Other bids are natural.
	
	\section[1C--2C]{\cl1--\cl2}
	
	\subsection{UPH}
	
	Either minor, GF. \di2 relays to ask suit: \he2 shows any hand with clubs, over which \sp2 is TaJ. With diamonds you immediately zoom into TaJ responses starting with \sp2 being the first step.
	
	\subsection{PH}
	
	Always diamonds, \di2 is TaJ

	\section{Other}
	\subsection{Balanced}	
		\di2, \he2 and \sp2 are balanced hands; similar to old \cl1--\sp1--1NT responses. The primary difference is the responses to bids are updated to use CIRKLE instead of controls. \rem{Anyone not using CIRKLE yet can retain the old structure.}
	
	\begin{bidtable}{\orauction{1c,2d}}
		\he2 & Agrees \heartsuit, asks CIRKLE. \\
		\sp2 & Shows 4+ spades, asks for support. 2NT by responder shows 4+ \spadesuit, over which \cl3 is CIRKLE. Bids above 2NT are suitless CIRKLE responses. \\
		2NT & Denies a major, suitless CIRKLE responses. \\
		\cl3--\di3 & Natural \\
		\he3 & No slam interest, 4 \heartsuit, choice of games. \\
		\sp3 & No slam interest, exactly 4 \spadesuit, COG.  Rarely used. \\
		3NT & To play \\
	\end{bidtable}
	
	\begin{bidtable}{\orauction{1c,2h}}
		\sp2 & Agrees \spadesuit, asks CIRKLE. \\
		2NT & Denies a fit, suitless CIRKLE \\
		\cl3--\he3 & Natural \\
		\sp3 & No slam interest, exactly 4 \spadesuit, COG \\
		3NT & To play \\
	\end{bidtable}
	\subsection{3 suiters}
	
	2NT thru \he3 are 3 suited hands with shortness in the bid suit. 2NT shows short spades. Over these 4x1 bids, every suit can be agreed below game. 3NT to play, agreeing a suit triggers CIRKLE for that suit.
	
	\subsection[3S]{\sp3}
	
	``Gambling'' type hand, AKQ 7th or better with nothing much on the side. Intended to be a picture bid. No special responses at this time. Probably should develop something, perhaps \cl4 asking for shortness. RKC/CIRKLE doesn't seem to make much sense, only shape seems likely to matter.

\end{document}




