\documentclass[tom-ari]{subfile}
\begin{document}
	
	\chapter{Defensive Bidding (They Open)}
	
	\section{General Guidelines}
	
	Overcalls can be light and can be 4 cards. The latter happens most often with length in the opening suit and the inability to make a takeout double due to shape (i.e., doubleton in a major). It often has opening strength or close to it.
	
	Takeout doubles can be mildly offshape. Doubletons in unbid minors are common. Doubletons in unbid majors are forbidden except for double-and-bid hands.
	
	Bidding tends to be for lead direction when light.
	
	2 level minor overcalls are often 6, occasionally 5 sound when no other call is reasonable. It is common to double instead of overcalling with \shape(5332).
	
	Direct NT overcalls are sound, (15+)16--18. In the sandwich position we bump this up by a point to a nominal range of (16+)17--19. Tom has made simple 1NT overcalls on bad 19 counts as well.
	
	If the opponents make an invitational or better artificial bid, our double is lead directing. (Example: 1M--\di3 Bergen).  If the artificial bid can be less than Inv values then our double is takeout.  (Example: 1M--\cl3 Bergen). Note for Tom: I think we agreed that this principle applies over Drury as well, even though I typically play double of Drury as lead directing. \rem{Verify? We could encapsulate the principle by saying this applies if it is our first turn to make a call. If we have previously passed then it is always lead directing.}
	
	After a jump overcall, NT bids are generally an attempt to show a secondary suit, especially one which would otherwise be awkward.  Example from play:  
	
	\auctionpart[,Ari,,Tom]{1c,2d,p,2s,p,2n}
	
	The 2NT bid here would show 4 \hhh{s}. 
	
	\subsection{Balancing}
	
	While most balancing actions after 1x are still natural, it is worth calling out some differences.
	
	Jumps are good hands and good suits, the higher the jump the better the hand. I recently balanced with 4M and had 8 good and a side Ace. That feels about right to me.
	
	1NT balance is wide ranging if they open a major, about 11--16 or so. Our \cl2 becomes a Size-ask Stayman, with normal responses showing 11--14 and 2NT showing any 15--16. Over the 2NT rebid, \cl3 is re-Stayman.
	
	1NT balance over a minor is 11--14 with little system. Cuebid is Stayman, no transfers. This auction is NF signoff:
	
	\auction{1d,p,p,1n,p,2c}  
	
	\section{Overcall Methods}
	
		Needs to be filled in, I noticed this wasn't here when I went hunting during practice.
	
	\section{Strong Club}
	
	Suction at all levels: Every bid is either the next higher (transfer) or the 2 remaining suits (2 lower). Double takes the place of the suit bid in the system; for example, (\cl1)--Dbl is \diamondsuit or (\heartsuit \& \spadesuit). In general, the higher you bid the more distribution you have. 
	
	We will often start with pass with a decent hand. Immediate bids tend to be destructive.
	
	NT bids show ``shape'': (\diamondsuit \& \spadesuit) or (\clubsuit \& \heartsuit).
	
	Responses in general are pass or correct. Raising the suit bid is a ``cuebid'' showing a good hand for the possible combinations and game interest. NT bids turn off the pass/correct signal; they generally ask overcaller to bid clubs over which all bids by advancer are natural, not pass/correct.
	
	If Overcaller bids NT after a pass/correct bid that shows a secondary 4 card suit with 6+ in the ``transfer'' suit. In other words, 1 suited by only kinda-sorta. This is 100\% optional on the part of overcaller.
	
	Suction is on directly over the \cl1 bid as well as any artificial response which does not indicate shape, only values.\footnote{Minor inferences about distribution are allowed. For example, a \he1 bid which shows 8--11 without 5 spades would be considered a suction eligible bid. A \he1 bid which is a transfer to spades would not.}
	
	\section{Polish or Balanced Club}
	
	No special methods, other than double can have any minor suit distribution.
	
	\rem{I think this is correct. I actually like Suction here too, but I seem to recall that you do not. When I do play Suction I play it constructive on the 1 level with double being takeout, not suction.}
	
	\section[Precision 1D]{Precision \di1}
	
	Regardless of promised length, we play \di2 as natural, \he2 as weak (NF) Michaels and \di3 as strong Michaels. 2NT is still \heartsuit \& \clubsuit. If we bid \di2 naturally, \clubsuit ~becomes the ``cuebid'' suit.
	
	As per balanced \cl1, takeout doubles are random with respect to minor suit distribution.
	
	\section[Transfer response to 1C]{Transfer response to \cl1}
	
	After (\cl1)--Pass--(1Red) xfer we play that double is a normal takeout double (\diamondsuit \& OM) and accepting the transfer is the weird takeout double (\clubsuit \& OM). 2 of the transfer suit is natural, just as the standard (1x)--P--(1M)--2M would be.
	
	After (\cl1)--Pass--(\sp1) it may depend on the meaning of \sp1. Most play that as diamonds (I think), in which case double is just takeout for the majors. I suspect that's a reasonable agreement for most \sp1 meanings but it is possible we may run across something which is worth having a separate agreement.
	
	\section{Kaplan Inversion}
	
	After (\he1)--Pass--(\sp1), double is a light spade overcall and 1NT is takeout. \sp2 is natural and sound, a hand that would have bid \sp2 over standard 1NT response.
	
	After (\he1)--Pass--(1NT), double is takeout for the minors.
	
	\section{1NT}
	
	Over their 1NT opener we play Woolsey:  \cl2 Majors, \di2 Multi, 2M M+min.
	
	Over strong NT, double is a 4 card major, longer minor. Over weak NT it is penalty when possible. (Passed Hand = ?)

	\rem{It may be worthwhile treating all \third seat 1NT openers as weak.}
	
	\begin{bidtable}{\compauct{1n,2c,p}}
		\di2 & Equal length in majors \\
		2M & NF \\
		2NT & Limit+ raise. \cl3 all min, \di3 equal length non-min, 3M shorter major non-min. \\
		3M & Mixed Raise \\
	\end{bidtable}

	\begin{bidtable}{\compauct{1n,2d,p}}
		2M & Pass or Correct \\
		2NT & Limit+ raise. \cl3 all min, 3Red = xfer. Over \cl3, 3M Pass or Correct \\
		3M & Pass or Correct, Mixed \\
		\cl4 & Please xfer to your suit \\
		\di4 & Please bid your suit \\
		4M & To play, \textit{not} pass or correct \\
	\end{bidtable}

	\begin{bidtable}{\compauct{1n,2M,p}}
		2NT & Asks minor, shows game interest \\
		\clubsuit & Pass or correct for minor any level \\
		\diamondsuit & Good raise of Maj, any level. \di4 sets up a forcing pass. \\
	\end{bidtable}

	Advances over \cl2/\di2 are different if the opps double the bid.
	
	\begin{bidtable}{\compauct{1n,2c,x}}
		Pass & \clubsuit, NF \\
		\di2 & \diamondsuit, NF \\
		XX & Equal length, overcaller to choose \\
		Other & No change \\
	\end{bidtable}

	\begin{bidtable}{\compauct{1n,2d,x}}
		Pass & \diamondsuit, NF \\
		XX & Bid your suit \\
		2M & Natural, not pass or correct \\
		Other & No change, including 3M bids still being Pass or Correct \\
	\end{bidtable}
	
	\section[2C Strong]{\cl2 Strong}
	
	Suction, as per Strong Club
	
	\section[2C Precision]{\cl2 Precision}
	
	Not sure if this is discussed. I like playing that \di2 is an artificial limited takeout, something like 9--13 with 2 or 3 suits. Dbl becomes 14+ takeout.
	
	\section{Multi}
	
	Since we are allowed to have a written defense to Multi in any event, I've decided to include the one generally given by the USBF to its teams for play abroad.  It's an enhanced version of the ``option 2'' defense.
	
	It is \textbf{several} pages long; it may take a bit to get it all typed up, but this is a start.
	
	\subsection{Direct Bids}
	
	\begin{bidtable}{\compauct{2d}}
		Dbl & 13-15 or 19+ Bal, or 18+ Unbal (13-16 if partner is a PH) \\
		\he2/\sss & Natural.  Cheapest bid in OM is \underline{natural} \\
		  & \he2--\sp3 is a cuebid.  \sp2--\he4 is a splinter. \\
		  & 3m response 4+ since no Q available. \\
		  & 4m jumps = splinter \textbf{except \sp2--\cl4 = bal or spl slam try.} \\
  	2NT & 16-18, resp as per 2NT opener.  \\
  		& If responder bids 3M P/C, Dbl (and delayed Dbl) are takeout, other natural. \\
 		\cl3 & Natural.  \di3 Artificial, like a cuebid. \\
 		\di3 & Natural.  \he3 2-way, see below. \\
 		3M & Strong, OM is a cuebid. \\
 		3NT & Based on long minor. \cl4 asks, LH. +1 over that is RKC. \\
 			& \di4/\hhh ~~Texas Xfer \\
		\cl4 & \ccc~\& Major, \di4 asks.  GF \\
		\di4 & \ddd~\& Major, \he4 P/C. GF \\
	\end{bidtable}

	\subsection{Overcalls}
	
	Over Dbl Pass or Correct by responder
	
	\begin{bidtable}{\compauct{2d,2h/\sss,x}}
		XX & Good Hand \\
		2NT & Running \\
		Suits & Natural \\
		P (P) & Opener should probably run \\
	\end{bidtable}
	
	\begin{bidtable}{\compauct{2d,3c,p,3d,p}}
		\he3 & Natural or \hhh~ stopper w/o a \sss~ stopper \\
			& \sp3~= 4 \hhh{s}, now 3NT = NF w/ \hhh~ stop but no \sss~ stop \\
			& 3NT \sss~stop, no interest in playing in \hhh \\
		\sp3 & Presumed Natural, F \\
		3NT & Natural w/ \sss~ stopper, \hhh~ stopper unknown \\
	\end{bidtable}

	\begin{bidtable}{\compauct{2d,3d,p}}
		\he3 & Natural or a \hhh~ stop and no \sss~ stop \\
			& \sp3 shows interest in playing in \hhh, 3NT by advancer shows the \hhh~stop w/o \sss~stop \\
			& 3NT \sss~ stop, no interest in \hhh \\
			& \he4 4 card \hhh, no \sss~stop, NF \\
		\sp3 & Natural, F \\
	\end{bidtable}

	\begin{bidtable}{\compauct{2d,Bid,2\sss/3x}}
		X & Cards \\
		\he4 & Natural, even over \he3 \\
	\end{bidtable}
	
	\subsection{\di2--X--P/XX}
	
	\begin{bidtable}{\compauct{2d,x,Pass/XX}}
		Pass & Willing to defend \di2 X or XX, might not be a big diamond hand \\
		     & Major Q after Pass = ask stop \\
    2M & Nat, Comp.  \di3 = Q, other NAT \\
    2NT & 5+ \ccc, requests \cl3 when Dbl is 13-15. Over \cl3, \di3 = GF 1 suiter in \ccc, other NAT GF \\
    \cl3 & GF Stayman.  \cl3--\di3--3M = worried about stopper in OM if P/XX was neutral, worry about \ddd stopper if P/XX showed \ddd \\
    \di3/\hhh & Xfer, inv+.  See below (p. \pageref{multixfer}) for continuations. \\
    3NT & 11--16 w/ \ddd stop \\
    4m & Strong invite \\
    4M & NF \\
    4NT & Straight Blackwood \\
	\end{bidtable}
	
	\begin{noted}[Choosing between Pass and 2NT with a weak hand:]
	Generally, avoid bidding 2NT since it will often get NT played from the wrong side.  If responder's P/XX showed \ddd, bid 2NT with 3325 or 6+ \ccc; if responder's P/XX was neutral, usually pass.
	\end{noted}

	\begin{bidtable}{\compauct{2d,x,Pass/XX,2n,p}}
		\cl3 & 13-15 bal \\
		\di3 & Art, 19+ Bal. Forcing to 3NT or higher. (\cl4 ok) \\
			& \he3 Waiting, might be weak. 3NT NF, \cl4 Nat GF, 4NT Quant. \\
		\he3 & Nat 18+ unbal, Forcing to 3NT or higher (\cl4 ok). \sp3=Q \\
		\sp3 & Nat 18+ unbal, Forcing to 3NT or higher (\cl4 ok) \he4=Q \\
		4M & Slam try \\
	\end{bidtable}
	
	\begin{bidtable}{\compauct{2d,x,P/XX,p,2\hhh/\sss}}
		Dbl & 19+ Bal, Takeout with Lebensohl \\
		2NT & 19--21 Bal, stop in opener's M. Respond as per 2NT opener. \\
		Other & Nat, 18+ \\
	\end{bidtable}

	\begin{bidtable}{\compauct{2d,x,P/XX,p,2\hhh/\sss,p,p}}
		Dbl & Values, 2+ cards in M, no clear direction \\
		3M & Spl \\
		Other & Nat \\
	\end{bidtable}

	\begin{bidtable}{
		\begin{auctionhead}
			\compauct{2d,x,p,p,xx\auctionnote{1}} 
			\explain{1. Specific Major}
		\end{auctionhead}
	}  
		P & Neutral \\
		Other & Nat, 19+ \\
	\end{bidtable}

	\begin{bidtable}{
		\begin{auctionhead}
			\compauct{2d,x,p,p,xx\auctionnote{1},p,2\hhh/\sss\auctionnote{2}} 
			\explain{1. Specific Major}
			\explain{2. Can be same or different from major shown by XX}
		\end{auctionhead}
	}
		Dbl & Needs definition \rem{Cards takeout?}
	\end{bidtable}

	\label{multixfer}
	\begin{bidtable}{
			\begin{auctionhead}
				\compauct{2d,x,P/XX/2\hhh/\sss,3d\auctionnote{1},p}
				\explain{1. Xfer to \hhh}	
			\end{auctionhead}
	}
		\he3 & 13-15, reject invite \\
		\sp3 & Q, choice of games \\
		4m,4NT & Nat, 19+ \\
		3NT,\he4 & 13-15, accept invite \\
		\sp4 & RKC for \hhh \\		
	\end{bidtable}	

	\begin{bidtable}{
		\begin{auctionhead}
			\compauct{2d,x,P/XX/2M,3h\auctionnote{1},p} 
			\explain{1. Xfer to \sss}			
		\end{auctionhead}
	}
		\sp3 & 13-15, reject invite \\
		4m,4NT & Nat, 19+ \\
		3NT/\sp4 & 13-15, accept invite \\
		\he4 & Cuebid, 18+, fit. \sp4--4NT is RKC \\
	\end{bidtable}

	\subsection{\di2--Dbl--\he2}
	
	This section is all about bids after \di2--Dbl--\he2 where \he2 is P/C.
	
	\begin{bidtable}{\compauct{2d,x,2h}}
		Pass & NF (Dbl of correction = T/O) \\
		Dbl & 10-12 Bal or Semi-Bal, some defensive tolerance \\
		\sp2 & NF, typically 5+ \\
		2NT & Semi-Puppet to \cl3, either signoff (6-9) minor, GF minor or bal with only \sss~stop \\
		\cl3 & GF Stayman, does not promise a major \\
		\di3/\hhh & Inv+ xfer. See above (p. \pageref{multixfer}) for follow ups. \\
		\sp3 & GF with short \hhh~and 0-3 \sss. \\
		3NT & 11-16, both Majors stopped \\
		4m & Strong Invite \\
		4M & Nat NF \\
		4NT & Straight Blackwood \\
	\end{bidtable}

	\begin{bidtable}{\compauct{2d,x,2h,p,P/2\sss}}
		Dbl & 19+ Bal, T/O w/ Lebensohl \\
		2NT & 19-21 Bal, \hhh stop; bid as over 2NT overcall of \he2 \\
	\end{bidtable}

	\begin{bidtable}{
		\begin{auctionhead}
			\compauct{2d,x,2h,2n\auctionnote{1},p} 
			\explain{1. Semi-puppet to \cl3}
		\end{auctionhead}
	}
		\cl3 & 13-15 Bal \\
		\di3 & Art, 19+ Bal, GF \\
		\he3 & Nat, 18+ Unbal; \sp3=Q, 4m Nat \\
		\sp3 & Nat, 18+ Unbal; \he4=Q, 4m Nat \\
		3NT & Nat, good hand with a long minor \\
		4m & Nat, 18+ Unbal \\
		4M & Slam try \\
	\end{bidtable}

	\begin{bidtable}{\compauct{2d,x,2h,2n,p,3c,p}}
		Pass & How to play \cl3 \\
		\di3 & NF, expected end of auction \\
		\he3 & GF \ccc \\
		\sp3 & GF \ddd \\
		3NT & Bal GF, only \sss~stop \\
	\end{bidtable}

	\begin{bidtable}{\compauct{2d,x,2h,2n,p,3d,p}}
		\he3 & \ccc, 8-9, slam interest \\
		\sp3 & \ddd, 8-9, slam interest \\
		3NT & NF \\
		4m & 12+ Nat; +1 RKC, 4NT Nat \\
		4NT & Nat, highly invititational \\
	\end{bidtable}

	\begin{bidtable}{\compauct{2d,x,2h,x,p}}
		\sp2 & 4 \sss, denies a maximum with a \hhh~stop \\
		2NT & NF min \\
		3m & NF min \\
		\he3 & Max w/ 4 \sss~\& \hhh~stop \textbf{OR} 19+ (Semi-)Bal w/ 4 \sss. \\
			& \sp3\rightarrow3NT, \he4\rightarrow\sp4 (bid more with 19+) \\
		\sp3 & Max, 0-3 \sss, no stop \textbf{OR} 20+ Bal (bid again) \\
		3NT & Max, 0-3 \sss, stopper, NF \\
		4m & Nat, 19+; +1 RKC, 4NT Nat \rem{Orig notes say \he4 not +1} \\ 
	\end{bidtable}

	\subsection{\di2--Dbl--\sp2}
	
	As in the previous section, \sp2 is P/C.
	
	\begin{bidtable}{\compauct{2d,x,2s}}
		Pass & NF, Dbl of correction is T/O \\
		Dbl & 10-12 Bal or Semi-Bal, some defensive tolerance \\
		2NT & Semi-puppet to \cl3: 6-9 minor NF, GF minor or Bal with only \hhh~stop \\
		\cl3 & GF Stayman, does not promise a major \\
		\di3/\hhh & Xfer \\
		\sp3 & GF with \sss shortness, 0-3 \hhh \\
		3NT & 11-15, both Majors stopped \\
		4m & Strong invite \\
		4M & Nat NF \\
		4NT & Straight Blackwood \\
	\end{bidtable}

	\begin{bidtable}{\compauct{2d,x,2s,p,p}}
		Dbl & 19+ Bal, T/O w/ Lebensohl \\
		2NT & 19-21 Bal, \sss stop; bid as over 2NT overcall of \sp2 \\
	\end{bidtable}

	\begin{bidtable}{\compauct{2d,x,2s,p,3h}}
		Dbl & 19+ Bal \\
		Other & Nat, 19+ \\
	\end{bidtable}

	\begin{bidtable}{
		\begin{auctionhead}
			\compauct{2d,x,2s,2NT\auctionnote{1},p}
			\explain{1. Semi-puppet to \cl3}
		\end{auctionhead}
	}
		\cl3 & 13-15 Bal \\
		\di3 & Art, 19+ Bal, GF \\
		\he3 & Nat, 18+ Unbal, GF; \sp3=Q, 4m Nat \\
		\sp3 & Nat, 18+ Unbal, GF; \he4=Q, 4m Nat \\
		3NT & Nat, good hand with a long minor \\
		4m & Nat, 18+ Unbal \\
		4M & Nat, Slam try \\
	\end{bidtable}


%%%%%%%%%%
	
	\section{Flannery}
	
	Vs. \di2 Flannery:
	
	\begin{bidtable}{\compauct{2d}}
		X & Bal 13--15 bal or 19+ any \\
		\he2 & 3 suited takeout \\
		\sp2 & Natural \\
		2NT & 16--18 bal \\
		3m & Natural \\
	\end{bidtable}

	Vs. \he2 Flannery:
	
	\begin{bidtable}{\compauct{2h}}
		X & 3 suit takeout or 19+ any \\
		\sp2 & Natural \\
		2NT & 16--18 bal \\
		3m & Natural \\
	\end{bidtable}

	For both options here, we don't have a bid to show both minors. Some play the \he2 takeout bid as either 2 (minors) or 3 suited, which might be worthwhile to explore.
	
	Kit had an interesting comment, he rejects the standard defense style and plays initial double as light takeout and pass-then-double as strong takeout with (I think?) the \he2 cuebid as Michaels. I'm not sure I agree with all of that, but having light vs sound takeout seems like a plus.
	
	See \url{https://bridgewinners.com/forums/read/intermediate-forum/defenses-to-flannery/}
	
	\section{Weak 2}
	
	Mostly normal stuff, but a few slightly different agreements.
	
	\subsection{Preferensohl}
	
	Modified Leb; after 2NT the doubler bids their preferred minor instead of auto-puppeting to \cl3. Other bids are as per Lebensohl with Fast Denies.
	
	Upside: get to a better 3m contract when advancer has no where to go.
	
	Downside: Forces us to \cl4 rarely when advancer only wants to play clubs. 
	
	\subsection{Soloway}

	Soloway over 2M--2NT (note, not over \di2):
	
	\begin{bidtable}{\compauct{2M,2n,p}}
		\cl3 & Puppet to \di3 for sign off anywhere \\
		\di3 & Xfer to other major, Inv+. Opener can accept by bidding game or Q. \\
		3M & Stayman \\
		3OM & Puppet for minor hands as per 2NT opener. \\
	\end{bidtable}

	\subsection{Leaping Michaels \& Direct Cuebids}
	
	We do play Leaping Michaels (GF). Over 2M openers, 4m is that minor and OM, 4M Q is both minors and stronger than 4NT. 
	
	Over \di2 weak, \cl4 is \clubsuit \& \heartsuit, \di4 is \clubsuit \& \spadesuit, \di3 is majors.
	
	2M--3M is a lighter Michaels. We are allowed to play a partscore. We do not play the stopper ask that is commonly played alongside Leaping Michaels.
		
	\section{Gambling 3NT}
	
	There are 2 common defenses to Gambling, I am not certain which we play. Calling this out so that we select an option.
	
	Option 1:  Shorter minor takeout. \cl4/\di4 is your shorter minor (defaulting to clubs with equal, although typically you can tell which is the solid suit). Other bids are natural.
	
	Option 2:  Woolsey. \cl4 both majors, \di4 1 major, 4M = M+minor.
	
	In both of these treatments, double is balanced strength / penalty and 4NT is undefined.
	
	\section{Other}
	
	Currently no special defense to \di2 Precision (short \diamondsuit) or (Mini)Roman. I know there are some out there, but not worrying about it for now.
	
	For most artificial preempts we follow the provided defense. For multi we use ``option 2'', where bids are natural and pass then double is takeout.
	
\end{document}