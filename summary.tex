\documentclass[tom-ari]{subfile}
\begin{document}
	
\chapter{System Summary}


''TaJ++ Precision'' % Probably can use a better name

Strong Club system with 0+ \di1.  No "Precision" \di2 opener.  Most 10 HCP hands opened NV, allowed to pass a 10 Vul. Lighter openers possible with shape.  

\section{Opening Bid Summary}

\begin{description}
	\item[\cl1] Strong, Forcing, Artificial.  Typically 16+ HCP unbal or 17+ bal
	\item[\di1] 0+ \diamondsuit, 10-15.  Denies 5 card major unless 6+ \diamondsuit.
	\item[\he1] 5+ \heartsuit, 10-15
	\item[\sp1] 5+ \spadesuit, 10-15
	\item[1NT] 14-16.  5 card major, 6 card minor, \shape{5422} common.
	\item[\cl2] 6+ \clubsuit, 10-15.  5 card suit possible in \third seat for lead direction.
	\item[\di2] 6+ \diamondsuit, (8)9-12.  5 card suit possible in \third seat for lead direction.
	\item[\he2/\sp2] 3-9, 5 card suit common NV
	\item[2NT] 22-23
	\item[3x] Natural, aggressive
	\item[3NT] Good Major preempt.  (Namyats-like)
	\item[4x] Natural, aggressive
\end{description}

No special agreements for opening bids 4NT and higher.

\section{General Principles}

\begin{itemize}
	\item Doubles unless otherwise defined are takeout
	\item All Strange Bids are Forcing (ASBAF).  A general guideline to cover the unknown.
	\item In an auction where we've committed to a certain level and the opps interfere (including double), pass is generally more encouraging than bidding to the forced the level.  A common example is a cuebid (raise) being doubled, then rebidding our trump suit.
	\item We ignore most doubles by the opponent, bidding retains their meaning.  One notable exception is \cl1--X
	\item Minimum responses to opening bids: while we pass \di1 freely up to 9 HCP, we follow more standard approaches to responding to 1M: respond with an Ace, a King and a 5 count or any 6 count.  Responding to 2m is a little different, passing is quite possible with 8 or so points, especially with no fit.  Even 10 or 11 counts are possible over \di2.
\end{itemize}	

\section{Relays}

\subsection{TaJ}
TaJ relay as it currently exists.  Used in both \cl1 auctions and 1M--[+2] auctions.

\begin{description}
	\item[Special] In auctions where responder is unlimited, first step shows extra values.  Next step repeats TaJ and mirrors the limited relay.
	\item[+1] \shape{54xx} Relay for \second suit LMH, then shape NLH.  Immediately ``zooming'' past the 2nd suit LMH relay shows LMH void and \shape{5440}
	\item[+2] \shape{55xx} or better.  Secondary suit is always equal or shorter.  Relay for \second suit LMH, then shortness LHB.
	\item[+3] \shape{64xx}.  Primary suit can be longer, secondary always 4.  Relay for \second suit LMH, then shortness LHB.
	\item[+4] 6+ card suit with shortness, denies 4 card side suit.  Relay for short suit LMH.
	\item[+5] \shape{5322}
	\item[+6] \shape{6322} or \shape{7222}.  This may also explode into further descriptive items, such as cuebids.  \rem{Or new relays...} 
\end{description}

\subsection{CIRKLE}

CIRKLE --- Carmichael's Improved Roman Keycard, Locating Everything (version 2.0)
 
\normalsize

\rem{Updated CIRKLE. KCs down to 2, using a mod 4 instead of mod 5. Hopefully more compressed to allow better space usage. This is closer to classical controls with KCs (K and Q) getting a bonus. Also, ABCD order tweaked from Game Order to Up-The-Line. (Noted in SQUARE)}

\begin{description}
	\item[What] Replacement for RKC
	\item[When] Typically after a suit is established but below game.
	\item[How] Using slam points (RKC=2, Other=1), broken down into 5 buckets (mod 4) 
\end{description}

CIRKLE is a asking bid about the typical slam cards:  the 5 ``Aces'' that make up the typical RKC responses and the other 4 cards of primary interest:  the Queen of trumps and the 3 side Kings.

At present, the KCs are worth 2 points each and the side cards are 1 each.  I'll call this your ``slam points''.

When CIRKLE is used, responder computes his slam points and responds in one of 4 steps, each of those steps representing the slam points modulo 4. This is similar to RKC, which uses a modulo 3 system.  (0/3, 1/4, 2/5).  For us, the steps are simply increasing:  0/4/8/12, 1/5/9/13, 2/6/10/14, 3/7/11. (Max is 14.)

While it may seem at first that having 4 possibilities is too many, the auction and your hand is going to immediately eliminate some of the possibilities.  The response to date in testing has always been no worse than 1 of 2.  While that too may seem daunting, we follow up CIRKLE with another asking sequence, SQUARE (see below).  That will uniquely identify the bucket as well as locate cards if they can't be seen.

At present we do not zoom from the last step directly into SQUARE, but that may be considered going forward.

This is noted elsewhere in the notes as well, but adding here for the relay summary: there are situations where the CIRKLE bids are tight for room.  The general rule is that CIRKLE is on when \cl4 is available as a bid to kick it off.  In those situations, any bids below 3NT are CIRKLE in ABCD order. \cl4 puppets to \di4 for any remaining CIRKLE bids (unless D is the only suit remaining, in which case \cl4 is CIRKLE D.) 

\di4 puppets to \he4 for sign off anywhere; bidding 4NT after the sign off puppet is an escape to RKC 1430. 

Direct game bids are natural and NF but forward going.

\subsection{SQUARE}

SQUARE --- Spiral Qbid Using Adaptive Relay Extensions

\normalsize

\rem{These notes are based on preliminary testing, the final form may still be different.}

\begin{description}
	\item[What] Series of asking bids/responses about holdings in all the suits
	\item[When] After CIRKLE
	\item[How] Stop/Go for each suit.  Even values are a ``stop'', odd values are a ``go'', reversed for singletons.  
\end{description}

SQUARE is an idea incorporated in a few other relay systems, notably the Moss/Grue agreements as well as a few other sources. 

The idea is after a hand has been limited some how in terms of values (HCP, controls, ZZs, whatever) you have a mechanism to locate the exact honors.  For us, the limiting mechanism is CIRKLE - we identify the total slam points for the hand and then locate the honors specifically.

The ``adaptive'' part of SQUARE is the order in which the suits are asked about is dynamic, based on what is known about responders length in suits.  Rather than having a strict LMH type order, the suits are ordered (ABCD) based on a few factors. Our priorities:

\begin{enumerate}
	\item Trump suit is always ``A''
	\item Longer suits come before shorter suits. Lengths can be implied rather than specific.
	\item Up The Line: \clubsuit \diamondsuit \heartsuit \spadesuit 
\end{enumerate}

I think for many partnerships employing these type methods the trump suit isn't necessarily known.  I also believe that they have a slightly different tie-breaker mechanisms.  \rem{Game order proved to be cumbersome for minors, we risked going past game too frequently. The ability to stop in 5 of a major makes up the line likely better. Worth testing.}

The ``spiral qbid'' part of SQUARE comes next.  After we have identified the suit order ABCD, the asker uses the next step to ask about the ``A'' suit.  Responder looks at the AKQ cards (for the ``A'' suit) or the AK cards (for ``BCD'') in the suit and responds with how many of the high honors do they have in an even/odd parity.  ``Even'', 0 or 2, is the Stop, bidding the very next step.  (We want to be able to slow the auction down if we get a zero response.)  If responder has an ``Odd'' response, this is a Go - they look at the B suit and do a Stop/Go evaluation of that.  A Stop would be bidding 1 above the next step (+2), a Go would have them look at the C suit, and so on.

After all the suits have been asked about the top 3 cards, we revert to asking about the Q for the BCD suits, then the J for the 4 suits preserving the ABCD order.  Yes is Go, No is stop.  (Still odd/even).  Room permitting, asking for 10s, etc.

One interesting aspect of SQUARE is how it Zooms.  Not only can responder ``Zoom'' after a Go response, but Asker can also ``Zoom'' the questions.  Skipping 1 step of square skips 1 suit in the ask.  This can be space preserving when Asker already knows the answer to the question.  The simplest example would be skipping a suit where asker held AKQ.

``Sign off'' bids are not asks. These are pretty obvious, but listed below in the special cases.  One special note, if 4Maj is available for sign off then 5Maj would still be a step for asking.  It's only the lowest bid at a level (Game, Slams) that acts as sign off.

A few special cases:  

\begin{itemize}
	\item When a suit is known to be exactly a singleton, the Go/Stop responses are inverted.  Generally speaking, having a high honor is less useful in a singleton than elsewhere, so the inverted response makes some sense. 
	\item When a suit is known to be a void, there are no asking bids for that suit. (ABC only)
	\item 3NT, 6NT, the cheapest level of the trump suit and slams in the trump suit are never asking bids.
	\item 6NT is a hard stop, never make a response above 6NT.  
	\item 5NT is also not an asking bid, it is a puppet to \cl6, then asker will place the contract.  Most commonly used if the presumed trump suit is not the actual final contract.
\end{itemize}

At present we need to have a specific asking bid launch into SQUARE.  We may incorporate a zoom into SQUARE coming from CIRKLE down the road.
	
\end{document}




