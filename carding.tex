\documentclass[tom-ari]{subfiles}
\begin{document}

\chapter{Leads}

\section{Opening vs Suits}

We generally lead 3rd/Low from length.  The only real exception is if we have supported partner's suit, we may lead a more attitude type approach. (High from Xxx, low from Hxx, etc.)

Our default honor leads are A from Ace/King. If we have promised 4+ cards in a suit, we revert to Rusinow honors. Note that transfers such as \di1--(\he1)--Dbl do promise a suit, but negative doubles such as \di1--(\sp1)--Dbl do not promise any one suit, even the other major.  Takeout doubles are presumed to not claim any suit.

\rem{Note:} I don't know if we currently have an agreement if we have \textit{both} shown 4+ cards in a suit. I presume the default rule would echo the standard treatment of having Rusinow be ``off'' by the secondary bidder, but I think that makes little sense for us.  I think it would be better to have both sides which have promised 4+ cards to play Rusinow, whichever one ends up on lead.

Signals to honor leads are generally attitude, but if attitude is known they may revert to suit preference.

Example: \di1--(P)--\he1--(Dbl)--\he2 \ldots whichever hand is on lead would lead Rus honors in \heartsuit. 

\section{Opening vs NT}

Our spot leads are typically 2nd/4th from 4+ cards, top of xxx or fewer. (American 2nd/4th).  As versus suits, we may vary from this in partner's suit. We may lead low from xxx in some situations to show length -- a good example would be after a unsupported weak 2. If dummy is winning the trick and 3rd hand gets to signal, signals are count if the winning card is Q or lower, attitude if K or higher. For this purpose, it is presumed dummy plays the lowest of touching cards.  (e.g., if dummy has KQx and calls ``K'', we would still consider dummy to be winning the Q and is therefore count.)

Honor leads are Rusinow from Q to 9, with 8 being a pivot card\footnote{Pivot card is the break point for Rusinow. 9 promises 10 or shortness, 8 might have the 9 or not.}. ``Shortness'' from a Rusinow perspective here is anything less than 4 cards, so we would still lead Q from QJx.  

Ace from AK is the default honor lead, with K being the ``power'' lead, asking for count or unblock. (e.g., KQT9). Signals to other honors are generally attitude about the honor below the led card.


\section{Middle of the Hand}
\chapter{Signal agreements}
\chapter{Examples from Play}

\lipsum[4]
	
\end{document}