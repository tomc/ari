\documentclass[tom-ari]{subfile}
\begin{document}
	
\chapter{3NT}
``Namyats'', good major preempt. 3NT is a forcing call. Typically responder bids one of \cl4 or \di4, but there are meanings for other bids which likely have not been seen by Ari before. I'm going to write them out, but they literally have never come up in any partnership I have ``agreed'' these bids since I came up with them 25 years ago.
	
	
\begin{bidtable}{\orauction{3NT}}
	\cl4 & Requesting partner transfer to their major. Cuebids and RKC. \\
	\di4 & Requesting partner bid their major. Cuebids and RKC. \\
	& \hrule \\
	& Herein lies doom. Everything below is a specific sort of asking bid/relay. \\
	& \hrule \\
	\he4 & Asking for suit and about control in the off major. Passable! \\
	\sp4 & Asking for \clubsuit ~control. \\
	4NT & Asking for \diamondsuit ~control. \\
	\cl5 & Asking for \clubsuit ~high card control only. \\
	\di5 & Asking for \diamondsuit ~high card control only. \\
\end{bidtable}

Over \he4, responses are alternating \heartsuit/\spadesuit ~with the first step being *Pass*, the groups being no control, sing/void, High Card control.

Over \sp4/4NT, responses are alternating \heartsuit/\spadesuit ~with the groups being No control, sing/void, HC control.

Over \cl5/\di5, as above with no sing/void groups.

Example auctions:

3NT--\he4--Pass would show \heartsuit ~with no \spadesuit ~control. This specific response lets us get out in 4M with no control in the other major. This is the most likely bid to make when responder can't actually tell which major opener has but still has slam interest.

3NT--\sp4 -- \cl5 would show \spadesuit ~with no \clubsuit ~control.

3NT--\cl5 -- \sp5 would show \heartsuit ~with a high card \clubsuit ~control.
	
\end{document}

