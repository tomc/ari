\documentclass[tom-ari]{subfile}
\begin{document}
	
\chapter{Interference Defense (We Open)}
	
	\section[1C]{\cl1}
	
		\subsection[2S and Below]{\sp2 and Below}
		
		Over direct interference below \sp2 we play transfers, every suit bid showing the next suit with 5+ cards and 5+ HCP (good 4 ok). Over spade bids by the opponents, NT becomes a transfer to \clubsuit. NT bids aren't well defined over non-spade bids by the opponent. 
		
		Notably, transfers into suits the opponent have shown are still natural. We take the general approach the opponents are extremely untrustworthy here, where psyches and misbids are so common. Therefore we just bid our hands without worrying about what they are showing.
		
		Doubles show values with no suit to show. We are allowed to have a 5 card suit that we don't wish to show, but generally this is a more balanced hand.
	
		It is worth noting that all of this applies over double as well. We assume that the auction will likely become more competitive and therefore do not try to relay and what not. 
		
		\subsection{2NT and higher}
		
		When the interence is at 2NT or higher, our bids revert to natural and GF. Double is values, either invitational (6--7) or a GF with no direction. Cuebids are generally 2 suited hands, GF. (Not available below 3NT, so stopper ask doesn't make much sense. Those hands start with double.)
	
	\section[1D]{\di1}
	
		\subsection{Low Level Interference}
	
		Over \di1 --(Dbl or \he1) we play a similar transfer based system.
		
		\begin{bidtable}{\compauct{1d,x}}
			XX & 4+ \heartsuit, any strength \\
			\he1 & 4--5 \spadesuit \\
			\sp1 & Balanced or both minors. Responder pulls 1NT to show minors. \\
			1NT & Single minor, competitive. \cl2 is pass or correct. \\
			\cl2/\di2 & Natural, forcing 1 round \\
			\he2 & 6+ \spadesuit, any strength \\
			\sp2 & Both minors, mixed strength \\
			2NT & Natural GF, rarely used. \\
		\end{bidtable}
	
		Bids over \he1 overcall are the same except for XX \& 2NT. There is no redouble to show hearts, but it isn't needed.	2NT is natural and invitational.	
		
		\subsection{Other suit overcalls}
		
		Fairly standard methods. Negative doubles (no upper limit, high level doubles are about strength more than shape)
		
		\subsection{1NT overcall}
		
		\rem{15 Apr 2020 -- Adding what I typically play, not sure if we have discussed.}
		
		``Reverse Capp'':
		
		\begin{bidtable}{\compauct{1d,1n}}
			Dbl & Penalty \\
			\cl2 & Single suited minor or Minor+Major 2 suiter \\
			\di2 & Both Majors \\
			\he2 & \heartsuit \\
			\sp2 & \spadesuit \\
		\end{bidtable}
	
		\subsection{Example From Play}	
	
	For now moved from unsorted.
	
	\hrule
		
	\auctionpart{1d,p,1n,2s,?}
	
	Opp interfered RW into our invitational sequence. There are lots of approaches here that are possible, but we decided to play:
	\begin{description}
		\item[Pass] Non-forcing. Expectation is the 1NT bidder will not reopen unless they have a 5 card suit. 
		\item \rem{With Jenni recently I had this auction as the 1NT bidder, I balanced double for takeout. She bid 2NT scrambling. That seems like a fine agreement, I don't see the need to force responder to pass.}
		\item[Dbl] Penalty
		\item[2NT] Mod. Lebensohl: Not a puppet, but instead the 1NT bidder bids their better minor. We may lose out when opener has only clubs, but we will win when they have a 2 suited hand.
		\item[3x] Nat GF
	\end{description}

	\hrule

After we make a support double, new suits are NF. Jump or Q to force.

	
	\section{1M}
	
	\section{1NT}
	
	\section{2m}
	
	\section{2M}
	
	\section{Other}
	
\end{document}