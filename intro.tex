\documentclass[tom-ari]{subfile}
\begin{document}

\chapter{Introduction}

Welcome to the latest attempt by Tom to use the \LaTeX~family of tools to try to make system notes.  

In this version, rather than using just my old macros from back in the day I am incorporating the Bridge Winners style file (with very slight modification).  I also have a miniature version of my old macros in bridge-mini.sty that I am including in the project.

I'm using subfiles for the first time, my hope is that having separate files for different chapters of the notes will make management a bit easier, especially using a git repo.  
My expectation is the ``to-do'' section will be a list of items that we noted at the table.  As things are added to the notes, we can remove them from the to-do list.  Perhaps not perfect, but it's a reasonable first pass at managing the project on the whole.

The latest version of this file should be available at \url{https://github.com/tomc/systemnotes/blob/master/tom-ari.pdf}

\section{Code Snippets}

Throw some text as a test

\orauction{1c,1s,1n,2c,2d}

You can reference bids such as \cl{1} \di{2} \he{3} or \sp{4} inline, or even cards such as \clubs{A} \diamonds{K} \hearts{Q} \spades{J}. This can be expanded to suit holdings such as \spades{AKxx}.

\shape{5332} any 5332 pattern
\exactshape{5332} 5 \spadesuit, 2 \clubsuit, 3-3 in reds.  This is from BW style, previously Tom used \handpat{(5332)} or \handpat{5332}.  (Note there is supposed to be a change of font - seems subtle in this version.) 

\rem{Comments which are expected to be removed in the ``production'' version.  Can be useful for development.}

\ari{Testing the Ari version.}

\rem{New!!}

I've added a new environment, bidtable.  It's similar to tabular so uses \& to separate fields and double backslash to end lines instead of the description/item environment.

Sample:

\begin{bidtable}{\orauction{1c,1d,7n\underline{}}}
	Pass & expected action \\
	XX & automatic if doubled \\
\end{bidtable}

\section{Notation}

\begin{description}
	\item[R] Simple Raise
	\item[R+1] One above a simple raise
	\item[DR] Double Raise
	\item[TR] Triple Raise
	\item[LMH] Low-Middle-High
	\item[LHB] Low-High-Both (Shortness relay after 10+ known cards.)
	\item[+1] Next Bidding Step
	\item[M] Major.  If one has been shown, it is the same one.
	\item[OM] Other Major.  After a major is shown.
	\item[m,om] Minor, other minor.
	\item[JS] Jump Shift
	\item[DJS] Double Jump Shift
\end{description}


\end{document}